\section{Use Cases}
    \subsection{Scenarios}
    \textbf{SCENARIO 1 - Educator creates a new account} \\
    Emanuele is a teacher of Computer Science at Politecnico di Milano and has just discovered a new app useful for enhancing software development skills, so he decides to download it. \\
    Due to the fact he is an educator he decides to create a new account, and so to sign-in into the app as an educator. \\
    After this, he starts to understand how the app works.\\
    
    \textbf{SCENARIO 2 - Student creates a new account} \\
    Matteo is a student of Computer Engineering at Unimore in Modena and has just discovered a new app useful for developing software engineering skills, so he decides to download it. \\
    He really wants to become a software engineer so he decides to create an account as a student \\
    After this, he starts to play with the basics of the application.\\

    \textbf{SCENARIO 3 - An educator or a student log-in in his account} \\
    Sara wants to use the new application called ``CodeKata'' she downloaded yesterday, so she decides to open it. \\
    As her first action, she logs in because she registered the day before. \\
    After this, she successfully can see the home.\\

    
    \textbf{SCENARIO 4 - Educator creates a new tournament}
    Mario, a software engineering teacher at Politecnico di Milano, knows about an app called CodeKata, which is useful to help students improve their software development skills. \\
    Thus he opens the application and, after signin-up, he decide to create a new tournament. 
    At the bottom there is a navbar with a central button called  ``New'', so he clicks on it. After this, the app shows a form to be filled with these fields: a text box asking the name of the tournament, some fields regarding the use of badges (elements in the form of rewards that represent the achievements of the individual students), namely the title of the badge, and one or more rules that must to be fulfilled to achieve the badge, all added selecting between a range of variables. 
    After compiling all the form, he can now complete the creation of the new tournament, and as a result the application will show to him the tournament interface.\\


    \textbf{SCENARIO 5 - Educator creates a new battle}\\
    Mario, a software engineering teacher, wants to add a new battle in his tournament. He opens the application and from the homepage he selects the tournament he had already created. Then he clicks on ``Add new battle'' and fills the information requested by the platform:
    a button ``Upload the code'' where he has to attach the code that needs to be completed by the students in a .zip format, another button with a text saying  ``Upload the tests'' where he has to attach the code that is needed to test the previous one, in a .zip format too, another button with a text saying  ``Upload the build automation scripts'' where he has to put the build automation scripts in a .zip format, a field asking the programming language used in the code, a field asking for a textual description of the software project, a field asking the maximum number of students per group, another one asking the minimum number of students per group, a field asking the registration deadline and a field asking the final submission deadline, and a checkbox asking if he wants to add additional configurations for scoring.
    Mario want to add these features, so he click on the checkbox and now he is asked by the system in percentage how much his evalutation should impact the final score.\\


    \textbf{SCENARIO 6 - Student subscribe to a tournament}\\
    Alessandro is a data scientist who wants to improve his skills in software development, so he decide to sign-up to this new application he just downloaded, CodeKata. 
    In order to reach his goals he starts searching among all the possible tournaments the app offers, in the Discovery section, and he found a new tournament in a programming language he doesn't know. So, he subscribe to this tournament. 
    As soon as he does it, the tournament appears in his home page, and clicking on it all the incoming battles and the ranking are displayed.\\


    \textbf{SCENARIO 7- Student joins a battle alone}\\
    Lorenzo wants to test the new CodeKata application he downloaded the previous week, so he opens the application, logs in and then starts to look for tournaments he is interested in. He finds out a tournament called ``Java with Leotta'', and since he want to improve his Java skills he decides to view this tournament.
    After opening it, he sees when the next battle will expire, and since the expiration date is the week after he decide to join the battle. 
    After joining the battle, this will be showed in his homepage after clicking on the corresponding tournament, associated with all the ranks. \\


    \textbf{SCENARIO 8 - Student joins a battle in a team}\\
    Maura wants to test the new CodeKata application he downloaded the previous week, so he opens the application, logs in and then starts to look for tournaments he is interested in. He finds out a tournament called ``Python with Prof. Lembo'', and since he want to improve his Python skills he decides to view this tournament.
    After opening it, he sees when the next battle will expire, and since the expiration date is the week after he decide to join the battle. Before joining the battle, he remembered he had few friends with his same aim, so he decides to tell to them about this application. 
    They both download the application, create an account, and decide to participate in team with Maura, so they subscribe to the same tournament as Maura did. 
    After this, Maura joins the battle as a team, inviting her friends. They accept the propose, and now their team is formed.
    After joining the battle, this will be showed in Maura's and her friends homepage after clicking on the corresponding tournament, associated with all the ranks. \\


    \textbf{SCENARIO 9 - Student pushes code on github repository}\\
    Francesco has written a solution and he wants to test it, so he makes a push on the github repository. After doing so he opens his homepage and opens the battle and opens his battle overview page. He sees that he has passed two more test cases and he is happy and he looks at the test cases he is not passing to decide on what to focus next. He also notices that now is rank in the battle has been updated. \\


    \textbf{SCENARIO 10 - Educator manually evaluates the battles}\\
    Michele is a meticulous teacher, who knows CodeKata, but doesn't trust it very much, so he decides to evaluate the battles in percentage by himself. In order to do it, when creating a new battle for one of the tournaments he created, he put a check on the box saying he will do manual evaluates of the codes given by the students who joined his battles.
    The correction is done after the expiration date for the submission of the solutions by the students. When finished, Michele fill a numerical field associated to each solution, and then the system calculate the new result based on the percentage chosen by the teacher when he created the battle. 
    For example, Michele wants to manually evalute the solution of Caterina, a student of him, and gives her 3 points out of 5, while she received 4 out of 5. When creating the battle, Michele chooses that his evaluation would weight 50\% on the final mark, so the final score for Caterina will be 4.5 out of 5.
    While Michele marks the solutions, a notification is sent to each component of the battle, updating rankings, badges and all the stats.\\

    
    \textbf{SCENARIO 11 - Educator invites another Educator in his tournament to create a battle }\\
    Giorgio is a teacher and want his students to solve a problem in Python, therefore he asks his colleague Tommaso to create a battle for his tournament as he teaches a Python beginner course. So Giorgio opens the application, opens his tournament page, and clicks on ``invite a collaborator'', then he searches Tommaso's username and sends him the request. Tommaso later opens the application and notice that he has Giorgio's tournament in the homepage, and now he can create battles in it. \\

    \textbf{SCENARIO 12 - A joining deadline of a battle expires}
    Maurizio joined a battle the day before its expiration date: as a result, the day after he receives the link for the GitHub repository he has to fork in order to partecipate to the battle from the CodeKata application, namely in the Homepage, under the tournament he is subscribed to. Thus, Maurizio successfully fork the repository and start to code in order to obtain the best result in the ranks.\\

    \textbf{SCENARIO 13 - Badges are assigned after the conclusion of a tournament}
    Patrizia has just finished the last battle of a tournament she was really interested in, and she is happy for having learned so much about coding in the field of software development. Since the corrections are both manual and automatized, she has to wait some days to see the final rankings. 
    For now she is sure she never had more than 2 test failed.
    After the manual correction of the codes, she discovers she is the fourth in the rank! The reason of this result is with a certain probability the time for completing the battles: she always waited until the last minute to send her results. 
    Notwithstanding with that, she receives a notification from the CodeKata applications saying she has earned a new badge with the following title: ``At most two fails'', regarding the total number of tests she passed. Happy for this, this new badge is now visible in her profile.\\

    \textbf{SCENARIO 14 - User want to know the rankings of an ongoing tournament}\\
    Mario, a student, has just signed up and wants to check the rankings an ongoing tournament as he wants to get an idea of how KataBattle works. Therefore he scrolls through the ongoings tournament, selects one, and starts looking at the rankings. He can see the personal score of all the students partecipating.\\

    \textbf{SCENARIO 15 - Student checks ranking of a battle he is participating to}\\
    Gianluca, a student, his competing in a battle, and he wants too see how the other teams are doing, to understand how is team is positioned compared to the others. Therefore he selects the tournament that contains the battle, than the battles, and then he scrolls though the rank.\\

    \textbf{SCENARIO 16 - Student looks at the badges obtained by another user}\\
    Filippo, a student, wants too see the badges obtained by his friend Luca, so he searches him using his username, then selects his profile. Now the application is displaying Luca's profile, which contains all the badges that Luca has obteined in all tournaments he has partecipated in.
    
    
    \subsection{Use cases}
    \begin{table}[h]
    \textbf{[UC1] - Educator sign-up}
    
      \centering
      \begin{tabular}{|p{3cm}|p{14cm}|}
        \hline
         Name & Educator sign-up \\
        \hline
        Actors & Educator \\
        \hline
        Entry Condition & The user has selected the registration as an educator \\
        \hline
        Event flow &  1 - The user inserts his name, surname, a username, the email,
        the password and a confirmation of it 
        
        2 - The user clicks on the ``Register'' button

        3 - The system saves the data

        4 - The system shows the login page
        \\
        \hline
        Exit condition & The educator is successfully registered in the system \\
        \hline
        Exceptions & 1 - The email is already registered
        
        1 - The username is already registered 
        
        1 - The password confirmation field doesn't match with the password field\\
        \hline
      \end{tabular}
      
    \end{table}

    \begin{table}[h]
    \textbf{[UC2] - Student sign-up}
    
      \centering
      \begin{tabular}{|p{3cm}|p{14cm}|}
        \hline
         Name & Student sign-up \\
        \hline
        Actors & Student \\
        \hline
        Entry Condition & The user has selected the registration as a student \\
        \hline
        Event flow &  1 - The user inserts his name, surname, a username, the email,
        the password and a confirmation of it 
        
        2 - The user clicks on the ``Register'' button

        3 - The system saves the data

        4 - The system shows the login page
        \\
        \hline
        Exit condition & The student is successfully registered in the system \\
        \hline
        Exceptions & 1 - The email is already registered
        
        1 - The username is already registered 
        
        1 - The password confirmation field doesn't match with the password field\\
        \hline
      \end{tabular}
      
    \end{table}

    \begin{table}[]
    \textbf{[UC3] - User log-in}
    
      \centering
      \begin{tabular}{|p{3cm}|p{14cm}|}
        \hline
         Name & User log-in \\
        \hline
        Actors & Students, Educators \\
        \hline
        Entry Condition & Clicked on the log-in button \\
        \hline
        Event flow &  1 - The user inserts his username and his password
        
        2 - The user clicks on the ``Login'' button

        3 - The system checks the data

        4 - The system shows the home page
        \\
        \hline
        Exit condition & The student is successfully logged in the system \\
        \hline
        Exceptions & 3 - The username is not registered into the system
        
        3 - The username is registered but the password is not correct 
        \\
        \hline
      \end{tabular}
      
    \end{table}

    \begin{table}[]
    \textbf{[UC4] - Educator creates a tournament}
    
      \centering
      \begin{tabular}{|p{3cm}|p{14cm}|}
        \hline
         Name & Educator creates a tournament \\
        \hline
        Actors & Educators \\
        \hline
        Entry Condition & The educator clicks on the ``New'' button on the bottom navbar \\
        \hline
        Event flow &  1 - The user inserts the name of the tournament
        
        2 - The user chooses whether add badges or not

        3 - If he decided to add badges, he can click on a plus button, and then he inserts the name of badge and sets the rules of it using a predefined set of variable and logical connective for each badge

        4 - The user clicks on the ``Done'' button

        5 - The system saves the new tournament
        \\
        \hline
        Exit condition & The educator can see the tournament page and decide to add new battles to his new tournament \\
        \hline
        Exceptions &
        3 - Incomplete rules are not accepted (tot\_attended\_battles = )
        \\
        \hline
      \end{tabular}
    \end{table}

    \begin{table}[]
    \textbf{[UC5] - Educator creates a battle}
    
      \centering
      \begin{tabular}{|p{3cm}|p{14cm}|}
        \hline
         Name & User Educator create a battle \\
        \hline
        Actors & Educator \\
        \hline
        Entry Condition & Educator clicks on new battle \\
        \hline
        Event flow &  1 - The educator attaches the project code, the test cases, build automation scripts, minimum and maximum number of people per group, registration deadline, final submission deadline  
        
        2 - If the educator decides to add additional information for scoring inserts a percentage indicating the weights it has on the final score
        \\
        \hline
        Exit condition & New battle is now visible and accessible in the tournament \\
        \hline
        Exceptions & 1 - the project code, test cases and build automation scripts do not compile
        
        1 - the registration date is not admissible
        
        1 - the submission date in not admissible
        
        1 - not all required data is inserted by the educator
        \\
        \hline
      \end{tabular}
      
    \end{table}

    \begin{table}[h]
      \textbf{[UC6] - Student subscribe to a tournament}
      
      \centering
      \begin{tabular}{|p{3cm}|p{14cm}|}
        \hline
        Name & Student subscribe to a tournament \\
        \hline
        Actors & Students \\
        \hline
        Entry Condition & The student is logged and is scrolling in the discovery section \\
        \hline
        Event flow &  1 - The user choose a tournament
        
        2 - The user click on it

        3 - The user click the ``Subscribe'' button
        \\
        \hline
        Exit condition & The system shows the new tournament in the corresponding on the homepage \\
        \hline
        Exceptions & 2 - The subscribing date of the tournament is expired  
        \\
        \hline
      \end{tabular}
    \end{table}

    \begin{table}[h]
    \textbf{[UC7] - Student joins a new battle alone}
    
      \centering
      \begin{tabular}{|p{3cm}|p{14cm}|}
        \hline
        Name & Student joins a new battle alone \\
        \hline
        Actors & Students \\
        \hline
        Entry Condition & The student is logged and clicks on a tournament in the discovery section \\
        \hline
        Event flow &  1 - The user choose a battle
        
        2 - The user click on the ``Join'' button

        3 - The user chooses \textit{alone} as the modality of joining

        4 - The system shows a confirmation message
        \\
        \hline
        Exit condition & The system shows the new battle in the corresponding tournament on the homepage with the associated ranks \\
        \hline
        Exceptions & 2 - The joining date of the battle is expired  
        \\
        \hline
      \end{tabular}
      
    \end{table}

    \begin{table}[h]
      \textbf{[UC8] - Student joins a new battle in team}
      
      \centering
      \begin{tabular}{|p{3cm}|p{14cm}|}
        \hline
        Name & Student joins a new battle in team \\
        \hline
        Actors & Students \\
        \hline
        Entry Condition & The student is logged and clicks on a tournament in the discovery section \\
        \hline
        Event flow &  1 - The user choose a battle
        
        2 - The user click on the ``Join'' button

        3 - The user chooses \textit{in team} as the modality of joining

        4 - The user chooses the name of subscribed people to address his invite to

        5 - After they accept the invite, the team is formed
        \\
        \hline
        Exit condition & The system shows the new battle in the corresponding tournament on the homepage of each team member with the associated ranks \\
        \hline
        Exceptions & 2 - The joining date of the battle is expired  

        5 - If they don't accept the invite, after the deadline no member will be joined to the battle
        \\
        \hline
      \end{tabular}
    \end{table}


    \begin{table}[h]
    \textbf{[UC9] - Student pushes code }

    
      \centering
      \begin{tabular}{|p{3cm}|p{14cm}|}
        \hline
        Name & Student pushes code \\
        \hline
        Actors & Student \\
        \hline
        Entry Condition &  Student is participating in a battle and the battle has started \\
        \hline
        Event flow &  1 - Student pushes the code in the repository
        
        2 - The system is notified

        3 - The test cases, build automation scripts and the code uploaded by the student are run together

        4 - The number of test passed is made visible to the student through the application

        5 - The rank of the student is updated\\
        \hline
        Exit condition &  The score of the student in the battle is updated \\
        \hline
        Exceptions & 
        \\
        \hline
      \end{tabular}
      
    \end{table}

      \begin{table}[h]
    \textbf{[UC10] - Educator manually evaluates the battles }

    
      \centering
      \begin{tabular}{|p{3cm}|p{14cm}|}
        \hline
        Name & Educator manually evaluates the battles \\
        \hline
        Actors & Educator \\
        \hline
        Entry Condition &  Educator chooses to mark the checkbox saying he would manually evaluate the solutions when creating the battle he wants to manually review \\
        \hline
        Event flow &  1 - The educator see a solution
        
        2 - The educator evaluate the solution

        3 - The educator fill a field requiring the mark

        4 - The system calculates the new result for the student or the team who provided that solution
        \\

        \hline
        Exit condition &  The rank and the badges of the student or of the students in a team are updated \\
        \hline
        Exceptions & 3 - The number required is expressed in term of x out of tot\_number\_of\_tests, if the assigned score overflow the maximum the system will block the insertion of the datum
        \\
        \hline
      \end{tabular}
      
    \end{table}

    \begin{table}[]
    \textbf{[UC11] - Educator authorizes another educator}
    
      \centering
      \begin{tabular}{|p{3cm}|p{14cm}|}
        \hline
         Name & Educator authorizes another educator \\
        \hline
        Actors & Educators \\
        \hline
        Entry Condition & Educator giving authorization has created a tournament \\
        \hline
        Event flow &  1 - The educator clicks on add collaborator
        
        2 - The educator searches the other educator by username and selects him

        3 - The tournament is added to the collaborator homepage
        \\
        \hline
        Exit condition & The educator who has become a collaborator can now see the tournament in his homepage and can add a new battle  \\
        \hline
        Exceptions & 
        \\
        \hline
      \end{tabular}
      
    \end{table}

    \begin{table}[]
    \textbf{[UC12] - A joining deadline of a battle expires}
    
      \centering
      \begin{tabular}{|p{3cm}|p{14cm}|}
        \hline
         Name & A joining deadline of a battle expires \\
        \hline
        Actors & Students \\
        \hline
        Entry Condition & The deadline for joining to a battle expires \\
        \hline
        Event flow &  1 - The system creates the GitHub repository for the battle
        
        2 - The system share the link of the repository through a notification sent in the tournament section of every participant
        \\
        \hline
        Exit condition & The students joined in the battle can see the repository and decide to fork it in order to start programming \\
        \hline
        Exceptions & 1 - The repository isn't created successfully by GitHub
        \\
        \hline
      \end{tabular}
      
    \end{table}

    \begin{table}[]
    \textbf{[UC13] - Badges are assigned after the conclusion of a tournament}
    
      \centering
      \begin{tabular}{|p{3cm}|p{14cm}|}
        \hline
         Name & Badges are assigned after the conclusion of a tournament \\
        \hline
        Actors & Students \\
        \hline
        Entry Condition & The tournament is finished \\
        \hline
        Event flow &  1 - The system checks for all the badges conditions defined by the educators when they created the tournament for all the students 
        
        2 - The system send assigns a badge to the students which have respected its rules
        \\
        \hline
        Exit condition & The badge of the student becomes visible for everyone \\
        \hline
        Exceptions &
        \\
        \hline
      \end{tabular}
      
    \end{table}

    \begin{table}[]
    \textbf{[UC14] - User looks at the ranking of a tournament}
    
      \centering
      \begin{tabular}{|p{3cm}|p{14cm}|}
        \hline
         Name & User looks at the ranking of a tournament \\
        \hline
        Actors & Students and Educators  \\
        \hline
        Entry Condition & User logins and selects a tournament \\
        \hline
        Event flow &  1 - The system shows the personal score of each participant in a table
        \\
        \hline
        Exit condition & The user can successfully see the rankings \\
        \hline
        Exceptions &
        \\
        \hline
      \end{tabular}
      
    \end{table}

    \begin{table}[]
    \textbf{[UC15] - Student checks ranking of a battle he is participating in}
    
      \centering
      \begin{tabular}{|p{3cm}|p{14cm}|}
        \hline
         Name & Student checks ranking of a battle he is participating in \\
        \hline
        Actors & Students \\
        \hline
        Entry Condition & Student logins and selects the battle \\
        \hline
        Event flow &  1 - The system shows the current score of each team in a table
        \\
        \hline
        Exit condition & The user can successfully see the ranking \\
        \hline
        Exceptions &
        \\
        \hline
      \end{tabular}
      
    \end{table}

    \begin{table}[]
    \textbf{[UC16] - Student checks the badges obtained by another student}
    
      \centering
      \begin{tabular}{|p{3cm}|p{14cm}|}
        \hline
         Name & Student checks the badges obtained by another student \\
        \hline
        Actors & Students \\
        \hline
        Entry Condition & Student logins \\
        \hline
        Event flow &  1 - The student selects searches the profile of another student

        2 - The student selects the profile of the other student
        
        \\
        \hline
        Exit condition & The user can successfully see the badges of the other student in his profile \\
        \hline
        Exceptions &
        \\
        \hline
      \end{tabular}
      
    \end{table}