\subsection{Document Structure}
This Requirements and Analysis Specification Document is composed of four major sections.

The first one is the \textbf{Introduction}, whose main objective is to introduce the reader to the domain of interest for the system to be developed, mainly using natural language to describe all the most fundamental actors and elements involved in the interactions between the system and the outer world.\\
The \textbf{Purpose} provides the definition of the main goals for the application.\\
The \textbf{Scope} is dedicated to reprocessing the original stakeholders' requirements in a new high-level description of the domain of interest that aims at being as unambiguous and clear as possible. All the most meaningful actors of the world in which the system lives are mentioned and their role explained. Besides, the interactions between these actors and the system are touched at a high-level to clarify what will come next in the document and to justify some of the design choices that are taken in the following paragraphs of the RASD. From the natural language description of the system, world and machine phenomena can be derived and in fact, they come immediately afterwards. These can be interpreted as a schematisation of the previous description in which only the events occurring in the domain of interest are presented (see  "The World and the Machine", by Michael Jackson, 1995 for more information).\\
In the \textbf{Definitions} subsection it is possible to find specifications on terminology and vocabulary terms used throughout the document, so that ambiguity shouldn't emerge from reading. From the table provided in this part, synonyms for words employed in the RASD are also listed.

The second major section of this document is the \textbf{Overall Description}. This part of the RASD has several goals, which are achieved thanks to its subsections.\\
The first one is dedicated to scenarios. The \textbf{Scenarios} subsection aims at validating the stakeholders' needs by illustrating concrete instances and examples of interactions with the system to be developed.\\
The \textbf{Domain Class Diagram} and \textbf{State Charts} are UML diagrams that provide a graphical visualization of the world of interest, consistently with the Introduction section.\\
In the \textbf{Product Functions} chapter, the functional requirements of the system are listed in a schematic way. This analysis derives as a consequence of all the previous sections, in which the world of interest has been described and accurately observed in order to understand what requirements the system should meet.\\
Finally, the \textbf{Assumptions, Dependencies and Constraints} subsection is dedicated to listing all the events and elements of the domain which are not under the system's control or which the system has some dependency over.

The third relevant part of this RASD is \textbf{Specific Requirements}, which is more concerned about turning the functional requirements listed in the Product Functions section into schematic and graphical representations.\\
The \textbf{External Interface Requirements} subsection deals with the interfaces and modes of interactions between the system and external users or other software products.\\
The \textbf{Functional Requirements} chapter offers a schematic view of the functional requirements listed in the Product Functions section, by means of use cases, use case diagrams and sequence diagrams. A mapping between these graphical representations and the associated requirements is also provided.\\
The part named \textbf{Design Constraints} specifies any constraint that the system has to respect when being developed, while the last section called \textbf{Software System Attributes} lists a series of qualities that the software to be implemented must have and the way to achieve them (for instance reliability, availability...).

The final section called \textbf{Alloy} provides the study of the system through Alloy, which is a tool for analyzing systems and seeing if they are designed correctly.











